\documentclass[11pt, a4paper, leqno]{article}
\usepackage{a4wide}
\usepackage[T1]{fontenc}
\usepackage[utf8]{inputenc}
\usepackage{float, afterpage, rotating, graphicx}
\usepackage{epstopdf}
\usepackage{longtable, booktabs, tabularx}
\usepackage{fancyvrb, moreverb, relsize}
\usepackage{eurosym, calc}
% \usepackage{chngcntr}
\usepackage{amsmath, amssymb, amsfonts, amsthm, bm}
\usepackage{caption}
\usepackage{mdwlist}
\usepackage{xfrac}
\usepackage{setspace}
\usepackage{xcolor}
\usepackage{url} 
\usepackage{subcaption}
\usepackage{minibox}
% \usepackage{pdf14} % Enable for Manuscriptcentral -- can't handle pdf 1.5
% \usepackage{endfloat} % Enable to move tables / figures to the end. Useful for some submissions.


\usepackage[
    natbib=true,
    bibencoding=inputenc,
    bibstyle=authoryear-ibid,
    citestyle=authoryear-comp,
    maxcitenames=3,
    maxbibnames=10,
    useprefix=false,
    sortcites=true,
    backend=biber
]{biblatex}
\AtBeginDocument{\toggletrue{blx@useprefix}}
\AtBeginBibliography{\togglefalse{blx@useprefix}}
\setlength{\bibitemsep}{1.5ex}
\addbibresource{refs.bib}





\usepackage[unicode=true]{hyperref}
\hypersetup{
    colorlinks=true,
    linkcolor=black,
    anchorcolor=black,
    citecolor=black,
    filecolor=black,
    menucolor=black,
    runcolor=black,
    urlcolor=black
}


\widowpenalty=10000
\clubpenalty=10000

\setlength{\parskip}{1ex}
\setlength{\parindent}{0ex}
\setstretch{1.5}


\begin{document}

\title{European Factor Stock-picking and Screener\thanks{Jonathan Willnow, University Bonn. Email: \href{mailto:jona.willnow@hotmail.de}{\nolinkurl{jona [dot] willnow [at] hotmail [dot] de}}.}}

\author{Jonathan Willnow}

\date{
    {\bf 10.03.2022}
    \\[1ex]
    \today
}

\maketitle


\begin{abstract}
    The focus of this project is to build an interactive tool that helps individual investors in their buying decisions of individual stock. 
    The project tackles several aspects to lead to better informed decisions: First, I deliberately omit the names of the stocks and report only the ticker symbol.
    This should help individuals to not bias themselves when tehy see an popular or unpopular stock since the decision should be purely based on the fundamentals.
    Secondly, apart from reporting standard metrics, I calculate the empirically proven Fama French Asset Pricing Factors. Thirdly, I come up with an on Score to score the stocks
    with respect to the asset pricing factors and reported metrics. This Score is not an ultimate assessment since it does not capture all necessary informations, 
    but investors can use it to get a first insight into a stock.
    This project is based on scraped stocks for Germany, Europe, North-America and Japan and collects the metrics used for assessment and calculation of the factors
    from finance.yahoo.com. The long term plan is to automate this project such that it serves as a reliable, automated way to screen stocks and improve stock-picking decisions.
\end{abstract}
\clearpage

\section{Introduction} % (fold)
\label{sec:introduction}
Stock market participation increased in the last years. While this is overall a positive development, many investors lack tools to base their investments on or are solely using their
gut feeling and intuition. While there are many great tools available (some which are really expensive), I could not find a free tool that allows a comparison of stocks based on the Fama French Asset Pricing Factors - so I 
decided to build it myself. As this is work in progress, it is not available right now for the broad audience (would require some up scaling of the website), but I shared the results in a developed Dash-App with a big German speaking 
stock picking community, called "Kleine Finanzzeitung". The Dash-App allows users to access my results and my research without having a proper background in programming and git.
The Dash-App can be found here: \url{https://stockpickingapp.herokuapp.com}



\section{Methodology}
The idea appears simple: There exist more than only one risk factor $\beta$,  that determines the
performance of a portfolio (here, portfolio is assumed to be a index/ market portfolio).
Fama and French came up with the Fama and French three factor model that includes
the size of firms, their book-to-market values, and their excess return on the market. In other words, 
the three factors used are $SMB$ (small minus big), $HML$ (high minus low), and $r_f$, the
market return return less the risk-free rate of return. Building up on their work, many other
factors have been researched and found like the Fama French Quality Factor and Conservative 
Asset factor or the Carhardt Momentum Factor. While the Momentum Factor finds strong 
empirical support like the other factors, this project so far is limited to the factors 
found by Fama and French. 
A selection of individual Stocks is available for Europe, Germany, Japan and North America
that was collected using several scrapers and validation methods. To obtain reliable metrics
the metrics are collected from https://finance.yahoo.com
using the functionality of the python packe urllib.requests. The so obtained metrics were then 
used to compute the introduced factors according to the work of Fama and French. To present the 
results and allow for easy filtering, I decided to deploy this simple Dash App.

To briefly summarize the content of \cite{FF3}, the authors claim that average returns show little relation 
to the $\beta$ found by Sharpe and Lintner, not to the inter temporal asset-pricing models of Breeden (1979) 
and other asset-pricing models. Following this, they look into other variables that might explain average returns,
 such as the market capitalization, leverage, price-to-earnings ratio, the book to market equity
 value and so on and study their joint roles together with the $\beta$. To summarize the findings, 
 $\beta$ has little information when used jointly with other variables. When used jointly, the market capitalization,
  book-to-market equity show high explanatory power and absorb the effect of the price-to-earnings ratio and the leverage.
  Therefore, these two variables together with $\beta$ form the Fama French Three Factor Model in which the intercept 
  is close to zero, meaning that there is only little left in the so called $\alpha$, which is 
  traditionally claimed by mutual fund managers to be there "alpha" - their out-performance over the market.
  Therefore, the Fama French Three Factor model almost fully explains all this out-performance, which 
  just is related to a set of factors used by the managers and not their skill, insights, vision or whatever.

  In \cite{FAMA20151}, the two authors developed two more factors, namely the Quality factor and the conservatism asset factor.
  The quality factor is calculated using all accounting numbers from the end of the previous fiscal year.
  It is defined by the annual revenues minus the cost of goods sold, interest expenses, selling, general, and administrative expenses divided by the book equity.
 The Conservative Asset Factor is defined as the ratio of total assets of a stock at the fiscal year end of $t-1$ and the total assets at fiscal year end of $t-2$.
 Adding these factors to the Fama French Three Factor Model, this 5 factors build the Fama French Five Factor Model that I used for this project to construct the score.
 The score uses as a weighting of the different five factors their historical out-performance over the market return, which is obtained from the Fama French 
 webiste \url{https://mba.tuck.dartmouth.edu/pages/faculty/ken.french/data_library.html} for all the different regions where data is available. 

\section{Code}
As this project was part of the course Effective Programming Practices for Economists by Prof. von Gaudecker, the corresponding repository 
that is available on github follows \citet{GaudeckerEconProjectTemplates}. It offers great help in structuring and publishing a project. 
The code for this repo can be found on \url{https://github.com/JonathanWillnow/european_factor_stockpicking_screener}. 
Since I will be searching for collaboraters in the future, I will keep it public.

\section{How to run the code}
Please refer to the README.md file or the corresponding part within the documentation for how to run the project and its tasks.

\section{Further Development and Outlook}
This is work in progress. This project is not finished and work in progress. After I started it for the course Effective Programming Practices for Economists, 
I became aware of all the possible extensions that I could develop and implement. 
As outlined, I want to make it Open Source and will try to find other coding and finance enthusiasts who want to work with me on this project.
Some things that are in my mind:
\begin{itemize}
    \item Risk analysis for the stocks using volatility, downside volatility
    \item Moving from Dash to a pure Flask . This would be faster and enables more possibilities    
    \item Scaling up the website (this would require me to pay fees to Heroku. As for now, the service I am using is free). This also includes using a cloud solution for hosting the data
    \item Adding more stocks! I was thinking about every major stock exchange in europe and also including more emerging markets stocks
    \item Detecting outliers and anomalies / better filtering of the results
    \item Fixing my Score:
    \begin{itemize}
        \item Right now, the score is subject to outliers that might be the result of wrong data or circumstances that can not easily identified
        \item For instance bancrupt companies show great scores since they have a low MC and a low P/B ratio
        \item I have to find a way to make the score robust to this problematic
    \end{itemize} 
    \item Back-testing my Score
    \item Improving general code quality and speed
\end{itemize}



\setstretch{1}
\printbibliography
\setstretch{1.5}




% \appendix

% The chngctr package is needed for the following lines.
% \counterwithin{table}{section}
% \counterwithin{figure}{section}

\end{document}
